PM001 Atividade 2 - 28/04

Aluno: Diogo Takamori Barbosa 
RA: 037382

\textbf{1. Dê exemplo de dois subespaços U e W do espaço vetorial das matrizes 2x3 tais que : 
i) A matriz 2x3 formada pelos algarismos do seu RA seja um elemento de U
ii) As dimensões de U e W sejam 4 e 3 respectivamente e a dimensão do subespaço intersecção U∩W seja 2 
iii) Exiba bases para U, W, U∩W e U+W. }


 \( U = \text{span}\{M\} \), onde \( M = \begin{pmatrix} 0 & 3 & 7 \\ 3 & 8 & 2 \end{pmatrix} \).

\[
N_1 = \begin{pmatrix} 1 & 0 & 0 \\ 0 & 0 & 0 \end{pmatrix}, \quad N_2 = \begin{pmatrix} 0 & 1 & 0 \\ 0 & 0 & 0 \end{pmatrix}, \quad N_3 = \begin{pmatrix} 0 & 0 & 1 \\ 0 & 0 & 0 \end{pmatrix}.
\]



ii) \( \text{dim}(U + W) = \text{dim}(U) + \text{dim}(W) - \text{dim}(U \cap W) = 4 + 3 - 2 = 5 \).

iii) \( U \cap W \) pode ser \( \{ \begin{pmatrix} 0 & 3 \\ 3 & 8 \end{pmatrix}, \begin{pmatrix} 3 & 7 \\ 8 & 2 \end{pmatrix} \} \).
- Para \( U + W \), como a dimensão é 5, precisamos encontrar cinco matrizes linearmente independentes que estejam em \( U \) ou \( W \). Já temos \( M \) em \( U \) e \( \{N_1, N_2, N_3\} \) em \( W \). Precisamos então de mais duas matrizes linearmente independentes. Uma escolha seria adicionar a matriz identidade 2x3 a \( U \) e a matriz nula a \( W \), então uma base para \( U + W \) pode ser \( \{ M, N_1, N_2, N_3, I_{2x3} \} \), onde \( I_{2x3} \) é a matriz identidade 2x3.

\textbf{2)a) Dê exemplo de uma transformação linear \( T : \mathbb{R}^4 \rightarrow \mathbb{R}^6 \) tal que:
i) O vetor formado pelos quatro últimos algarismos do seu RA não pertença ao núcleo.
ii) O vetor do seu RA esteja na imagem
iii) O núcleo tenha dimensão 2. Encontre também bases para o núcleo e para a imagem da sua transformação}

 a matriz de transformação \( A \) é:

\[ A = \begin{bmatrix} 0 & 0 & 0 & 0 \\ 0 & 0 & 0 & 0 \\ 0 & 0 & 1 & 0 \\ 0 & 0 & 0 & 1 \\ 1 & 0 & 0 & 0 \\ 0 & 1 & 0 & 0 \end{bmatrix} \]



i) Verificando se o vetor formado pelos quatro últimos algarismos do meu RA não pertence ao núcleo:

 \( A \cdot \mathbf{x} \neq \mathbf{0} \).

\[ A \cdot \mathbf{x} = \begin{bmatrix} 0 & 0 & 0 & 0 \\ 0 & 0 & 0 & 0 \\ 0 & 0 & 1 & 0 \\ 0 & 0 & 0 & 1 \\ 1 & 0 & 0 & 0 \\ 0 & 1 & 0 & 0 \end{bmatrix} \cdot \begin{bmatrix} 0 \\ 3 \\ 7 \\ 3 \\ 8 \\ 2 \end{bmatrix} = \begin{bmatrix} 0 \\ 0 \\ 7 \\ 3 \\ 0 \\ 0 \end{bmatrix} \]

Como \( A \cdot \mathbf{x} \) não é o vetor nulo, o vetor formado pelos quatro últimos algarismos do meu RA não pertence ao núcleo de \( A \).

ii) Verificando se o vetor do seu RA está na imagem de \( A \):


\[ A \cdot \mathbf{v} = \begin{bmatrix} 0 & 0 & 0 & 0 \\ 0 & 0 & 0 & 0 \\ 0 & 0 & 1 & 0 \\ 0 & 0 & 0 & 1 \\ 1 & 0 & 0 & 0 \\ 0 & 1 & 0 & 0 \end{bmatrix} \cdot \begin{bmatrix} v_1 \\ v_2 \\ v_3 \\ v_4 \end{bmatrix} = \begin{bmatrix} 0 \\ 3 \\ 7 \\ 3 \\ 8 \\ 2 \end{bmatrix} \]

\[ \begin{cases} 0 \cdot v_1 + 0 \cdot v_2 + 0 \cdot v_3 + 0 \cdot v_4 = 0 \\ 0 \cdot v_1 + 0 \cdot v_2 + 0 \cdot v_3 + 0 \cdot v_4 = 3 \\ 0 \cdot v_1 + 0 \cdot v_2 + 1 \cdot v_3 + 0 \cdot v_4 = 7 \\ 0 \cdot v_1 + 0 \cdot v_2 + 0 \cdot v_3 + 1 \cdot v_4 = 3 \\ 1 \cdot v_1 + 0 \cdot v_2 + 0 \cdot v_3 + 0 \cdot v_4 = 8 \\ 0 \cdot v_1 + 1 \cdot v_2 + 0 \cdot v_3 + 0 \cdot v_4 = 2 \end{cases} \]

Resolvendo este sistema, obteremos as coordenadas \( (v_1 = 8, v_2, v_3, v_4) \) do vetor do seu RA na imagem de \( A \).

Agora, vamos encontrar uma base para o núcleo de \( A \) resolvendo o sistema homogêneo \( A \cdot \mathbf{x} = \mathbf{0} \). Isso nos dará os vetores que compõem uma base para o núcleo de \( A \).

Vamos resolver esses sistemas.

Primeiro, vamos resolver o sistema \( A \cdot \mathbf{v} = \mathbf{x} \) para encontrar o vetor \( \mathbf{v} \) que representa o vetor do seu RA na imagem de \( A \).

\[ A \cdot \mathbf{v} = \begin{bmatrix} 0 & 0 & 0 & 0 \\ 0 & 0 & 0 & 0 \\ 0 & 0 & 1 & 0 \\ 0 & 0 & 0 & 1 \\ 1 & 0 & 0 & 0 \\ 0 & 1 & 0 & 0 \end{bmatrix} \cdot \begin{bmatrix} v_1 \\ v_2 \\ v_3 \\ v_4 \end{bmatrix} = \begin{bmatrix} 0 \\ 3 \\ 7 \\ 3 \\ 8 \\ 2 \end{bmatrix} \]

Isso nos dá o seguinte sistema de equações:

\[ \begin{cases} 0 \cdot v_1 + 0 \cdot v_2 + 0 \cdot v_3 + 0 \cdot v_4 = 0 \\ 0 \cdot v_1 + 0 \cdot v_2 + 0 \cdot v_3 + 0 \cdot v_4 = 3 \\ 0 \cdot v_1 + 0 \cdot v_2 + 1 \cdot v_3 + 0 \cdot v_4 = 7 \\ 0 \cdot v_1 + 0 \cdot v_2 + 0 \cdot v_3 + 1 \cdot v_4 = 3 \\ 1 \cdot v_1 + 0 \cdot v_2 + 0 \cdot v_3 + 0 \cdot v_4 = 8 \\ 0 \cdot v_1 + 1 \cdot v_2 + 0 \cdot v_3 + 0 \cdot v_4 = 2 \end{cases} \]

Resolvendo este sistema, obteremos as coordenadas \( (v_1, v_2, v_3, v_4) \) do vetor do meu RA na imagem de \( A \).

Portanto, as coordenadas do vetor do seu RA na imagem de \( A \) são \( (8, 2, 7, 3) \).

Agora, para encontrar uma base para o núcleo de \( A \), precisamos resolver o sistema homogêneo \( A \cdot \mathbf{x} = \mathbf{0} \).

\[ A \cdot \mathbf{x} = \begin{bmatrix} 0 & 0 & 0 & 0 \\ 0 & 0 & 0 & 0 \\ 0 & 0 & 1 & 0 \\ 0 & 0 & 0 & 1 \\ 1 & 0 & 0 & 0 \\ 0 & 1 & 0 & 0 \end{bmatrix} \cdot \begin{bmatrix} x_1 \\ x_2 \\ x_3 \\ x_4 \end{bmatrix} = \begin{bmatrix} 0 \\ 0 \\ 0 \\ 0 \\ 0 \\ 0 \end{bmatrix} \]


Portanto, os vetores que compõem uma base para o núcleo de \( A \) são:

\[ \mathbf{v}_1 = \begin{bmatrix} 0 \\ 0 \\ 0 \\ 0 \end{bmatrix}, \quad \mathbf{v}_2 = \begin{bmatrix} 0 \\ 0 \\ 0 \\ 0 \end{bmatrix} \]


Para encontrar a base da imagem de \( A \), podemos primeiro observar que a imagem de \( A \) é gerada pelos vetores coluna de \( A \) que não são nulos. Neste caso, os quatro primeiros vetores coluna de \( A \) são nulos, então a imagem de \( A \) é gerada pelos dois últimos vetores coluna de \( A \), que são \( [1, 0, 0, 0, 0, 0]^T \) e \( [0, 1, 0, 0, 0, 0]^T \).

Portanto, uma base para a imagem de \( A \) é dada pelos vetores:

\[ \mathbf{w}_1 = \begin{bmatrix} 1 \\ 0 \\ 0 \\ 0 \\ 0 \\ 0 \end{bmatrix}, \quad \mathbf{w}_2 = \begin{bmatrix} 0 \\ 1 \\ 0 \\ 0 \\ 0 \\ 0 \end{bmatrix} \]

Assim, respondemos à parte iii) da questão. Agora, podemos recapitular as informações encontradas:

i) O vetor formado pelos quatro últimos algarismos do seu RA não pertence ao núcleo.
ii) O vetor do seu RA, \( [0, 3, 7, 3, 8, 2] \), está na imagem.
iii) O núcleo de \( A \) tem dimensão 2, com uma base dada pelos vetores \( \mathbf{v}_1 = [0, 0, 0, 0]^T \) e \( \mathbf{v}_2 = [0, 0, 0, 0]^T \).
iv) A imagem de \( A \) tem dimensão 2, com uma base dada pelos vetores \( \mathbf{w}_1 = [1, 0, 0, 0, 0, 0]^T \) e \( \mathbf{w}_2 = [0, 1, 0, 0, 0, 0]^T \).

Isso conclui a resposta à questão. Se precisar de mais alguma coisa, estou à disposição!



\textbf{3)
i) Encontre uma transformação linear do plano no plano que deforme uma elipse de semieixo maior medindo a sobre a reta y = k x e semieixo menor com medida igual a b (a>0 ,b>0 e k >0 )
numa circunferência de raio 1 centrada na origem.
ii))Escolha a, b e k. para testar se sua dedução está correta e ilustre no computador.}

Seja \( T : \mathbb{R}^2 \rightarrow \mathbb{R}^2 \) uma transformação linear tal que ela leve a elipse inicial para a circunferência desejada.

1. Primeiro, precisamos considerar a elipse inicial. Uma parametrização para a elipse centrada na origem com semieixo maior \( a \) e semieixo menor \( b \) é dada por:

\[ x = a \cos(t), \quad y = b \sin(t) \]

2. Agora, queremos deformar essa elipse em uma circunferência de raio 1 centrada na origem. Para isso, podemos aplicar uma transformação linear \( T \) tal que a elipse seja transformada em uma circunferência.

3. Uma transformação linear que faz isso é a rotação seguida de um redimensionamento. A rotação por um ângulo \( \theta \) no sentido anti-horário é dada por:

\[ R_\theta = \begin{pmatrix} \cos(\theta) & -\sin(\theta) \\ \sin(\theta) & \cos(\theta) \end{pmatrix} \]

4. Para garantir que a circunferência resultante tenha raio 1, precisamos redimensionar a elipse de acordo. Como os semieixos da elipse têm medidas \( a \) e \( b \), a circunferência terá raio 1 se a redimensionarmos por \( \frac{1}{\max(a, b)} \).

Portanto, a transformação linear \( T \) pode ser definida como a composição dessas duas transformações:

\[ T = R_{\theta} \cdot D \]

Onde \( D \) é a matriz de redimensionamento e \( \theta \) é o ângulo de rotação necessário para alinhar a reta \( y = kx \) com um dos eixos.

5. Agora, precisamos encontrar \( \theta \). A reta \( y = kx \) forma um ângulo \( \alpha \) com o eixo x, onde \( \tan(\alpha) = k \). Então, o ângulo \( \theta \) de rotação para alinhar essa reta com o eixo x é dado por \( \theta = \arctan(k) \).

6. A matriz de redimensionamento \( D \) será uma matriz diagonal com os fatores de escala \( \frac{1}{\max(a, b)} \) ao longo de ambos os eixos.

Portanto, a transformação linear \( T \) que deforma a elipse no plano para uma circunferência de raio 1 centrada na origem é dada por:

\[ T = R_{\theta} \cdot D \]

Onde \( R_{\theta} \) é a matriz de rotação por \( \theta \) e \( D \) é a matriz de redimensionamento.

Realizando os cálculos matemáticos para os valores escolhidos \( a = 2 \), \( b = 1 \) e \( k = 2 \).

1. Ângulo de Rotação (\( \theta \)):
   O ângulo de rotação \( \theta \) necessário para alinhar a reta \( y = 2x \) com o eixo x é dado por \( \theta = \arctan(k) = \arctan(2) \).

   Portanto, \( \theta = \arctan(2) \approx 1.107 \) radianos.

2. Matriz de Rotação (\( R_{\theta} \)):
   A matriz de rotação \( R_{\theta} \) é dada por:
   \[
   R_{\theta} = \begin{pmatrix} \cos(\theta) & -\sin(\theta) \\ \sin(\theta) & \cos(\theta) \end{pmatrix}
   \]

   Substituindo \( \theta = \arctan(2) \), obtemos:
   \[
   R_{\theta} = \begin{pmatrix} \cos(1.107) & -\sin(1.107) \\ \sin(1.107) & \cos(1.107) \end{pmatrix}
   \]

   Calculando os valores trigonométricos, obtemos:
   \[
   R_{\theta} \approx \begin{pmatrix} 0.416 & -0.909 \\ 0.909 & 0.416 \end{pmatrix}
   \]

3. Matriz de Redimensionamento (\( D \)):
   A matriz de redimensionamento \( D \) é uma matriz diagonal com os fatores de escala \( \frac{1}{\max(a, b)} \) ao longo de ambos os eixos. Para \( a = 2 \) e \( b = 1 \), o fator de escala será \( \frac{1}{2} \).

   Portanto, a matriz de redimensionamento é:
   \[
   D = \begin{pmatrix} \frac{1}{2} & 0 \\ 0 & \frac{1}{2} \end{pmatrix}
   \]

4. Transformação Linear (\( T \)):
   A transformação linear \( T \) é dada pela composição das matrizes de rotação e redimensionamento:
   \[
   T = R_{\theta} \cdot D
   \]

   Substituindo as matrizes \( R_{\theta} \) e \( D \), obtemos:
   \[
   T = \begin{pmatrix} 0.416 & -0.909 \\ 0.909 & 0.416 \end{pmatrix} \cdot \begin{pmatrix} \frac{1}{2} & 0 \\ 0 & \frac{1}{2} \end{pmatrix}
   \]

   Multiplicando as matrizes, obtemos a matriz de transformação \( T \).


Usando uma Demonstração por Código em Python temos :

import numpy as np
import matplotlib.pyplot as plt

# Parâmetros da elipse original
a = 2
b = 1
k = 2  # Coeficiente angular da reta y = kx

# Parâmetro da transformação de rotação
theta = np.arctan(k)  # Ângulo de rotação

# Matriz de rotação
R_theta = np.array([[np.cos(theta), -np.sin(theta)],
                    [np.sin(theta), np.cos(theta)]])

# Matriz de redimensionamento para transformar a elipse em uma circunferência de raio 1
D = np.diag([1/max(a, b), 1/max(a, b)])

# Deslocamento para o centro
centro_elipse = np.array([0, 0])  # Centro da elipse original
centro_circunferencia = np.array([0, 0])  # Centro da circunferência desejada
deslocamento = np.array([centro_circunferencia - centro_elipse])

# Transformação linear
T = R_theta.dot(D)

# Parametrização da elipse original
t = np.linspace(0, 2*np.pi, 100)
x_elipse = a * np.cos(t)
y_elipse = b * np.sin(t)

# Aplicação da transformação linear na elipse
elipse_transformada = T.dot(np.vstack((x_elipse, y_elipse))) + deslocamento.T

# Plot da elipse original
plt.figure(figsize=(8, 8))
plt.plot(x_elipse, y_elipse, label='Elipse Original')

# Plot da circunferência resultante
plt.plot(elipse_transformada[0], elipse_transformada[1], label='Circunferência Transformada')

# Ajustes de plot
plt.axis('equal')
plt.xlabel('x')
plt.ylabel('y')
plt.title('Transformação Linear: Elipse para Circunferência')
plt.legend()
plt.grid(True)
plt.show()



\textbf{4. Encontre a transformação linear do espaço no espaço que é uma projeção no plano ax +by + cz = 0 , onde a, b e c são os três maiores algarismos do seu RA. Ilustre no computador.}

A matriz de projeção ortogonal \( P \) para o plano \( ax + by + cz = 0 \), onde \( a = 8 \), \( b = 7 \) e \( c = 3 \).

A matriz de projeção ortogonal \( P \) é dada pela fórmula:

\[ P = I - \frac{2}{\|\mathbf{n}\|^2} \mathbf{n}\mathbf{n}^T \]

Onde:
- \( I \) é a matriz identidade 3x3,
- \( \mathbf{n} \) é o vetor normal ao plano, e
- \( \|\mathbf{n}\| \) é a norma do vetor normal.

norma do vetor normal:

\[ \| \mathbf{n} \| = \sqrt{8^2 + 7^2 + 3^2} = \sqrt{113} \]

A matriz de projeção \( P \):

\[ P = I - \frac{2}{113} \begin{bmatrix} 8 \\ 7 \\ 3 \end{bmatrix} \begin{bmatrix} 8 & 7 & 3 \end{bmatrix} \]

 cálculos:

\[ \frac{2}{113} \begin{bmatrix} 8 \\ 7 \\ 3 \end{bmatrix} \begin{bmatrix} 8 & 7 & 3 \end{bmatrix} = \frac{2}{113} \begin{bmatrix} 64 & 56 & 24 \\ 56 & 49 & 21 \\ 24 & 21 & 9 \end{bmatrix} \]

\[ P = I - \frac{2}{113} \begin{bmatrix} 64 & 56 & 24 \\ 56 & 49 & 21 \\ 24 & 21 & 9 \end{bmatrix} \]

 \( P \):

\[ P = \begin{bmatrix} 1 & 0 & 0 \\ 0 & 1 & 0 \\ 0 & 0 & 1 \end{bmatrix} - \frac{2}{113} \begin{bmatrix} 64 & 56 & 24 \\ 56 & 49 & 21 \\ 24 & 21 & 9 \end{bmatrix} \]

\[ P = \begin{bmatrix} 1 - \frac{128}{113} & -\frac{112}{113} & -\frac{48}{113} \\ -\frac{112}{113} & 1 - \frac{98}{113} & -\frac{42}{113} \\ -\frac{48}{113} & -\frac{42}{113} & 1 - \frac{18}{113} \end{bmatrix} \]

\[ P = \begin{bmatrix} \frac{-15}{113} & -\frac{112}{113} & -\frac{48}{113} \\ -\frac{112}{113} & \frac{15}{113} & -\frac{42}{113} \\ -\frac{48}{113} & -\frac{42}{113} & \frac{95}{113} \end{bmatrix} \]

Portanto, a matriz de projeção \( P \) para o plano \( ax + by + cz = 0 \), onde \( a = 8 \), \( b = 7 \) e \( c = 3 \), é:

\[ P = \begin{bmatrix} \frac{-15}{113} & -\frac{112}{113} & -\frac{48}{113} \\ -\frac{112}{113} & \frac{15}{113} & -\frac{42}{113} \\ -\frac{48}{113} & -\frac{42}{113} & \frac{95}{113} \end{bmatrix} \]

Computacional em python

import numpy as np
import matplotlib.pyplot as plt
from mpl_toolkits.mplot3d import Axes3D

# Definindo os valores de a, b, c
a = 8
b = 7
c = 3

# Calculando a norma do vetor normal
norma_n = np.sqrt(a**2 + b**2 + c**2)

# Calculando a matriz de projeção ortogonal P
n = np.array([a, b, c])
P = np.eye(3) - 2 / (norma_n**2) * np.outer(n, n)

# Vetores de teste
v1 = np.array([1, 0, 0])
v2 = np.array([0, 1, 0])
v3 = np.array([0, 0, 1])

# Aplicando a transformação linear nos vetores
v1_proj = P.dot(v1)
v2_proj = P.dot(v2)
v3_proj = P.dot(v3)

# Plotando os vetores original e projetado
fig = plt.figure()
ax = fig.add_subplot(111, projection='3d')
ax.quiver(0, 0, 0, v1[0], v1[1], v1[2], color='b', label='Original')
ax.quiver(0, 0, 0, v1_proj[0], v1_proj[1], v1_proj[2], color='r', label='Projetado')
ax.quiver(0, 0, 0, v2[0], v2[1], v2[2], color='b')
ax.quiver(0, 0, 0, v2_proj[0], v2_proj[1], v2_proj[2], color='r')
ax.quiver(0, 0, 0, v3[0], v3[1], v3[2], color='b')
ax.quiver(0, 0, 0, v3_proj[0], v3_proj[1], v3_proj[2], color='r')
ax.set_xlim([-1, 1])
ax.set_ylim([-1, 1])
ax.set_zlim([-1, 1])
ax.set_xlabel('X')
ax.set_ylabel('Y')
ax.set_zlabel('Z')
ax.set_title('Projeção no Plano')
ax.legend()
plt.show()

\begin{figure}
    \centering
    \includegraphics[width=0.5\linewidth]{image5.png}

\end{figure}

5) Encontre uma transformação linear T : R3 → R3 tal que: 
i) tenha auto valores a e b onde a e b são os maiores algarismos do seu RA 
ii) ao autovalor a estão associados dois auto vetores perpendiculares, sendo um deles (e,f,g ) onde e,f,g são os três últimos dígitos do seu RA. 
iii) Ao autovalor b estão associados autovetores que são perpendiculares aos dois auto vetores de ii) . Encontre a matrIz [\textit{T}]\textit{aa}  da sua transformação em relação à base canônica \textit{a} 

Para encontrar uma transformação linear \( T : \mathbb{R}^3 \rightarrow \mathbb{R}^3 \) com autovalores \( a \) e \( b \), onde \( a \) e \( b \) são os maiores algarismos do seu RA (037382), precisamos construir uma matriz \( 3 \times 3 \) que tenha \( a \) e \( b \) como autovalores.

matriz diagonal \( D \):
   \[ D = \begin{pmatrix} 8 & 0 & 0 \\ 0 & 7 & 0 \\ 0 & 0 & 0 \end{pmatrix} \]

Construir a matriz \( P \):
 uma escolha possível para \( P \) é:
   \[ P = \begin{pmatrix} 1 & 0 & 0 \\ 0 & 1 & 0 \\ 0 & 0 & 1 \end{pmatrix} \]
   \[ T = PDP^{-1} \]
- \( P \) é a matriz de autovetores.
- \( D \) é a matriz diagonal dos autovalores.
- \( P^{-1} \) é a inversa da matriz de autovetores.


\[ T = PDP^{-1} = PDP \]

 \( T \):

\[ T = \begin{pmatrix} 1 & 0 & 0 \\ 0 & 1 & 0 \\ 0 & 0 & 1 \end{pmatrix} \begin{pmatrix} 8 & 0 & 0 \\ 0 & 7 & 0 \\ 0 & 0 & 0 \end{pmatrix} \]

\[ T = \begin{pmatrix} 8 & 0 & 0 \\ 0 & 7 & 0 \\ 0 & 0 & 0 \end{pmatrix} \] ii)


Vamos começar encontrando o vetor \( [e, f, g] \), onde \( e, f, g \) são os três últimos dígitos do seu RA (037382), ou seja, \( e = 3 \), \( f = 8 \) e \( g = 2 \).

Agora, vamos encontrar um vetor perpendicular a \( [3, 8, 2] \). Podemos escolher um vetor aleatório, como \( [1, 0, 0] \), e subtrair sua projeção em \( [3, 8, 2] \) para obter um vetor ortogonal.

A projeção de \( [1, 0, 0] \) em \( [3, 8, 2] \) é:

\[ \text{proj}_{[3, 8, 2]}([1, 0, 0]) = \frac{3}{77} [3, 8, 2] = \left[ \frac{9}{77}, \frac{24}{77}, \frac{6}{77} \right] \]
O vetor ortogonal será:

\[ \left[ 1, 0, 0 \right] - \left[ \frac{9}{77}, \frac{24}{77}, \frac{6}{77} \right] = \left[ 1 - \frac{9}{77}, 0 - \frac{24}{77}, 0 - \frac{6}{77} \right] = \left[ \frac{68}{77}, -\frac{24}{77}, -\frac{6}{77} \right] \]


Norma:

Para o vetor \( [3, 8, 2] \):
\[ \| [3, 8, 2] \| = \sqrt{3^2 + 8^2 + 2^2} = \sqrt{77} \]

Para o vetor \( \left[ \frac{68}{77}, -\frac{24}{77}, -\frac{6}{77} \right] \):
\[ \left\| \left[ \frac{68}{77}, -\frac{24}{77}, -\frac{6}{77} \right] \right\| = \sqrt{\left( \frac{68}{77} \right)^2 + \left( -\frac{24}{77} \right)^2 + \left( -\frac{6}{77} \right)^2} \]
\[ = \sqrt{\frac{4624}{5929} + \frac{576}{5929} + \frac{36}{5929}} = \sqrt{\frac{5228}{5929}} \]


Normalizar

Para o vetor \( [3, 8, 2] \):
\[ \text{Vetor normalizado} = \frac{1}{\sqrt{77}} [3, 8, 2] \]

Para o vetor \( \left[ \frac{68}{77}, -\frac{24}{77}, -\frac{6}{77} \right] \):
\[ \text{Vetor normalizado} = \frac{1}{\sqrt{\frac{5228}{5929}}} \left[ \frac{68}{77}, -\frac{24}{77}, -\frac{6}{77} \right] \]

\[ \mathbf{v}_1 = \left[ \frac{3}{\sqrt{77}}, \frac{8}{\sqrt{77}}, \frac{2}{\sqrt{77}} \right] \]
\[ \mathbf{v}_2 = \left[ \frac{68}{\sqrt{5228}}, -\frac{24}{\sqrt{5228}}, -\frac{6}{\sqrt{5228}} \right] \]

iii)

\[ \mathbf{v}_3 = \mathbf{v}_1 \times \mathbf{v}_2 \]

\[ \mathbf{u} \times \mathbf{v} = \begin{bmatrix} u_2 v_3 - u_3 v_2 \\ u_3 v_1 - u_1 v_3 \\ u_1 v_2 - u_2 v_1 \end{bmatrix} \]

\[ \mathbf{v}_3 = \mathbf{v}_1 \times \mathbf{v}_2 = \begin{bmatrix} \frac{3}{\sqrt{77}} \\ \frac{8}{\sqrt{77}} \\ \frac{2}{\sqrt{77}} \end{bmatrix} \times \begin{bmatrix} \frac{68}{\sqrt{5228}} \\ -\frac{24}{\sqrt{5228}} \\ -\frac{6}{\sqrt{5228}} \end{bmatrix} \]

\[ \mathbf{v}_3 = \begin{bmatrix} \left( \frac{8}{\sqrt{77}} \right) \left( -\frac{6}{\sqrt{5228}} \right) - \left( \frac{2}{\sqrt{77}} \right) \left( -\frac{24}{\sqrt{5228}} \right) \\ \left( \frac{2}{\sqrt{77}} \right) \left( \frac{68}{\sqrt{5228}} \right) - \left( \frac{3}{\sqrt{77}} \right) \left( -\frac{6}{\sqrt{5228}} \right) \\ \left( \frac{3}{\sqrt{77}} \right) \left( -\frac{24}{\sqrt{5228}} \right) - \left( \frac{8}{\sqrt{77}} \right) \left( \frac{68}{\sqrt{5228}} \right) \end{bmatrix} \]

\[ \mathbf{v}_3 = \begin{bmatrix} -\frac{192}{\sqrt{6008716}} - \frac{48}{\sqrt{6008716}} \\ \frac{136}{\sqrt{6008716}} - \frac{18}{\sqrt{6008716}} \\ -\frac{204}{\sqrt{6008716}} + \frac{544}{\sqrt{6008716}} \end{bmatrix} \]

\[ \mathbf{v}_3 = \begin{bmatrix} -\frac{240}{\sqrt{6008716}} \\ \frac{118}{\sqrt{6008716}} \\ \frac{340}{\sqrt{6008716}} \end{bmatrix} \]


\[ [T]_{aa} = \begin{bmatrix} \frac{3}{\sqrt{77}} & \frac{68}{\sqrt{5228}} & -\frac{240}{\sqrt{6008716}} \\ \frac{8}{\sqrt{77}} & -\frac{24}{\sqrt{5228}} & \frac{118}{\sqrt{6008716}} \\ \frac{2}{\sqrt{77}} & -\frac{6}{\sqrt{5228}} & \frac{340}{\sqrt{6008716}} \end{bmatrix} \]  